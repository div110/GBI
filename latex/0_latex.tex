\documentclass[12pt,a4paper,twoside,]{article}
\usepackage[inner=4cm,outer=2.5cm,top=2.5cm,bottom=2.5cm]{geometry}
\usepackage{setspace}
\usepackage{graphicx}
\usepackage{titlesec}
\usepackage{hyperref}
\usepackage{tablefootnote}
\onehalfspacing

\setlength{\parskip}{6pt}



\usepackage{fancyhdr}
\renewcommand{\headrulewidth}{0pt}
\pagestyle{fancy}
\fancyhf{}
\fancyfoot[LE,RO]{\thepage}

\setlength{\headheight}{14.5pt}

\usepackage{fontspec}
\setmainfont{TeX Gyre Termes}

\begin{document}
\begin{figure}[h!]
\centering
\includegraphics[scale=0.5,]{obrazky/sssi_logo.png}


	
\textsf{\textbf{\Huge Maturitní práce\\}}
\textsf{ \textbf{\Huge Script pro instalaci operačního systému Gentoo \\}}
\textsf{ \Huge \\Studijní obor: 18-20-M/01 \\Informační technologie\\}
\textsf{\LARGE \\Autor: Halamka David\\
Třída: IT4/A\\
Vedoucí práce: Petr Martínek
}
\vspace*{7em}
\textsf{\LARGE\\
Datum odevzdání: 31. března 2025 }
\thispagestyle{empty}
\end{figure}


\newpage
\thispagestyle{empty}
Prohlašuji, že jsem předkládanou práci vypracoval sám za použití zdrojů a~literatury v~ní uvedených. Souhlasím s~tím, aby moje maturitní práce byla využívána pro potřeby Střední školy spojů a~informatiky, Tábor, Bydlinského 2474 nebo jiných subjektů, které se podílely na zadání práce.\\ 
\\	…………………….~~~~~~~~~~~~~~~~~~~~~~~~~~~~~~~~~~……………………..\\
datum~~~~~~~~~~~~~~~~~~~~~~~~~~~~~~~~~~~~~~~~~~~~~~~~~~~~~~~~~~podpis



\newpage
\renewcommand{\contentsname}{\textsf{Obsah}}\tableofcontents

\newpage
\section{\textsf{Anotace}}
{Teoretická část maturitní práce popisuje proces instalace operačního systému Gentoo
	a~její vliv na tvorbu skriptu. V~práci je zároveň rozvedena problematika stěžejních
	kroků instalace a~částí operačního systému. Cílem práce bylo usnadnit novým uživatelům instalaci a~konfiguraci
	systému pro jednodušší přechod z~jiných operačních systémů. Z~toho důvodu je skript volně přístupný na webových stránkách Github. 
	\vspace{2cm}
	\\Klíčová slova: Linux, Bash, Operační systém, Gentoo
	%\\Keywords:
}


\newpage

\section{\textsf{Příprava disků}}
{Příprava disků je nutným krokem každé instalace. Hlavním důvodem je, 
že stávající schéma rozdělení a~souborové systémy nemusí být v~souladu s~novým operačním systémem. 
Z toho důvodu nestačí disky pouze smazat. Použité schéma rozdělení je GPT. 
Mezi hlavní důvody pro výběr GPT patří: lepší kompatibilita s~UEFI systémy, 
vyšší maximální počet oddílů, podpora disků větších než 2TB a~kontrola integrity.
Oddíly a~jejich nastavení mají velký vliv na paměť i~výkon systému. Proto je potřeba, 
aby skript i~samotná instalace přizpůsobily rozdělení specifikacím systému.} 

{První je EFI oddíl. Slouží pro zavádění UEFI systémů. Bude obsahovat bootloader. 
Velikost je pevně nastavena na 1GB, jelikož není potřeba škálovat se systémem.
Oddíl bude naformátován jako fat32.}

{Druhý je Swap oddíl. Swap obsahuje části operační paměti, vyhodnocené jako nej-\\méně používané. 
Rozbor samotného algoritmu pro správu paměti je mimo záběr maturitní práce. Je třeba,
aby velikost Swap oddílu rostla s~kapacitou paměti RAM. Tento vztah je způsoben vlastnostmi hibernace, 
která nahraje celý obsah operační paměti na Swap oddíl. Má-li tedy systém využívat hibernaci je nutné,
aby velikost oddílu byla ostře větší, než velikost paměti RAM. }

{Třetí je Root oddíl. Obsahuje komplement EFI oddílu. Je to zároveň také jediný oddíl z~trojice, 
u kterého lze účinně zvolit souborový systém. Zabírá zpravidla zbytek místa na disku.}

{Skript vytváří oddíly pomocí nástroje \textit{fdisk}. \textit{fdisk} je interaktivní nástroj na tvorbu \\oddílů, proto je nutné, aby skript dodával instrukce přímo do vstupu nástroje, ne shellu. Přesměrování provádí \textit{Here document}, který dokáže přesměrovat dávkovaný vstup přímo do programu.\\ Výběr disku:}

\texttt{select diskname in /dev/*; do\\}
\texttt{if [[ -n \$diskname ]]; then\\}
\texttt{\hspace*{2em}echo "Zvolen: \$diskname"\\}
\texttt{\hspace*{2em}break\\}
\texttt{else\\}
\texttt{\hspace*{2em}echo "Vyberte zařízení"\\}
\texttt{fi\\}
\texttt{done\\}

\hspace{-1.5em}\textit{select} umožňuje vybrat soubor ze složky \textit{/dev}, kde se nacházejí bloková zařízení.
\newpage
{Pro škálování Swapu je potřeba zjistit informace o~velikosti paměti RAM. Ty jsou uložené v~souboru \textit{/proc/meminfo}. Pro \textit{fdisk} je potřeba, aby jsme získali pouze nume-\\rický údaj. Přečtení a~zpracování tedy vypadá následovně:\\
	
\texttt{swapsize=\$(sudo cat /proc/meminfo | head -n 1 | \\\hspace*{1.5em}sed 's/[\^{}0-9]//g')}\\

\hspace*{-1.5em}Výsledné číslo pak zvětšíme o~50~\%:\\ 

\texttt{swapsize=\$(((\$swapsize/1000000)*3/2))}\\

\hspace{-1.5em}Ovládání pro \textit{fdisk} je blíže rozepsáno v~tabulce níže.
\begin{table}[h]
	\begin{tabular}{|c|c|c|}\hline
		Vstup & Zkratka & Popis \\ \hline
		-&g&Vytvoří nový GUID Partition table. \\ \hline
		n&noauto-pt&Vytvoří nový oddíl bez nové tabulky oddílů. \\ \hline
		t&type&Následuje číslo, přiřadí oddílu typový štítek. \\ \hline
		w&wipe&Fyzicky přemaže aktuální rozdělení disku.\tablefootnote{Někdy také označován jako 'write'}
		\\ \hline
		i&info&Vypíše informace o oddílu. \\ \hline
		l&list&Vypíše všechny typy oddílů. \\ \hline
		v&verify&Ověří tabulku rozdělení. \\ \hline
		
	\end{tabular}
\end{table}




\newpage
\subsection{\textsf{Souborový systém}}\hypertarget{Souborový systém}{}
Manuální instalace i~skript dávají uživateli možnost výběru souborového systému pro kořenový oddíl.
Souborový systém slouží k~interpretaci dat ve formě souborů, adresářů a~metadat. Dělí se na dvě hlavní skupiny: Žurnálovací a~Copy on Write.\\
Žurnálovací systémy uchovávají záznam prováděných operací ve speciální datové struktuře v~rámci jejich oddílu. 
Záznam slouží k~obnově konzistence souborového systému v~případě přerušení zápisu, způsobeným například odpojením zdroje elektřiny. \\
Copy on Write systémy nevytváří kopii souboru pro úpravu, pouze předají odkaz na již existující soubor a~to klidně více procesům najednou.
Kopii dat vytvoří až v~případě, kdy se nějaký z~procesů pokusí přepsat data. Tento přístup umožňuje integraci snapshotů a~RAIDů.
Skript vybírá mezi třemi souborovými systémy: XFS, btrfs a~ext4. \\Je vhodné zmínit, že formátování odstraní veškerá data z~oddílu. 
Formátování oddílů pro souborový systém ve skriptu provádí příkaz mkfs: 

\texttt{mkfs.vfat -F 32 /dev/sda1}\\
\texttt{\hspace*{1.5em}mkfs.\$filesystem /dev/sda3}

\hspace*{-1.5em}Speciální případ je Swap, který je potřeba prvně inicializovat: 

\texttt{mkswap /dev/sda2}

\hspace*{-1.5em}A poté aktivovat:

\texttt{swapon /dev/sda2}

\hspace*{-1.5em}Nyní stačí namontovat oddíly na jejich adresáře:

\texttt{mount /dev/sda3 /mnt/gentoo}\\
\texttt{\hspace*{1.5em}mount /dev/sda1 /mnt/gentoo/efi}

\newpage
\section{\textsf{Stage Tarball}}
{Pro instalaci základního systému je třeba tzv. Stage 3~Tarball, neboli archiv obsahující
minimální systém pro začátek instalace. Archivů je mnoho a~dělí se podle profilů.

Každý profil blíže specifikuje budoucí záměr a~funkce systému. Například profil
‘Desktop’ obsahuje balíčky, respektive jejich USE přepínače, usnadňující zprovoznění uživatelského rozhraní.
Profilů je velké množství, čehož důsledkem je více funkčních kombinací. Z~ tohoto důvodu skript umožňuje zvolit archiv jak automaticky, tak i~ručně pomocí URL adresy. 

Pro výchozí výběr byl zvolen archiv z~FTP serveru Fakulty informatiky
Masarykovy Univerzity v~Brně. Skript řeší výběr archivu pomocí jednoduché logiky,
která upřednostní manuální výběr, je-li dostupný:}\\

{\hspace*{+1.5em}\texttt{read -p stage3 2>\&1}\\
{\hspace*{+1.5em}\texttt{if [ -z "\$stage3" ]; then}\\
{\hspace*{+1.5em}\texttt{\hspace*{+1.5em}wget https://ftp.fi.muni.cz/pub/linux/gentoo/...\\}
{\hspace*{+1.5em}\texttt{else}\\
{\hspace*{+1.5em}\texttt{\hspace*{+1.5em}wget \$stage3}\\
{\hspace*{+1.5em}\texttt{fi}\\

\hspace{-1.5em}{Nesprávně zvolený profil lze po dokončení instalace změnit pomocí:\\}

\texttt{eselect profile list}\\
\texttt{\hspace*{--1.5em}eselect profile set \#}\\

\hspace{-1.5em}{Uživatel by však měl mít na paměti, že změna profilu může v~některých případech
vést až k~překompilování celého systému.}\\

\hspace{-1.5em}V tuto chvíli se doporučuje, aby uživatel provedl kontrolu archivu pomocí checksumů.
Ty lze získat ze souboru .DIGESTS z~jakéhokoliv zrcadla. Pro druhý, vlastní checksum, je doporučený nástroj \textit{sha256sum}, který vytváří checksum pomocí algoritmu sha256. Použití:

\texttt{sha256sum nazevSouboru}

\hspace{-1.5em}Z bezpečnostních důvodů není doporučeno používat starší algoritmy.\footnote{Například md5, sha1...}

\newpage
\subsection{\textsf{Instalace archivu}}
Před instalací je vhodné nastavit systémový čas. K~tomu slouží \textit{chrony}:\\
\hspace*{1.5em}\texttt{chrony -q}\\
 anebo, není-li dostupné RTC, pomocí: \\ \hspace*{1.5em}\texttt{date mmddhhmmyy}.\\
 
V případě jakýchkoliv pochybností je možné datum zkontrolovat pomocí \textit{date}.
\\Archiv je potřeba extrahovat do cílového adresáře nového systému. Proto je nutné,\\
aby byl adresář již předem vytvořený. Ve velké většině případů \textit{/mnt/gentoo}.\\
Tarball instalujeme pomocí nástroje tar v~příkazu:

\texttt{tar xpvf stage3-*.tar.xz --xattrs-include='*.*' \\\hspace*{1.5em}--numeric-owner -C /mnt/gentoo}\\

\hspace{-1.5em}Pro lepší orientaci bude význam přepínačů a~argumentů rozepsán:\\

\texttt{x} - Soubor bude extrahován.\\
\hspace*{1.5em}\texttt{p} - Ponechat oprávnění. \\
\hspace*{1.5em}\texttt{v} - Zapíná výpis průběhu extrakce.\\
\hspace*{1.5em}\texttt{f} - Určuje, jaký soubor má být extrahován a cestu k němu.\\
\hspace*{1.5em}\texttt{--xattrs-include='*.*'} - Zachová rozšiřující atributy.\\
\hspace*{1.5em}\texttt{--numeric-owner} - Zachová vlastnictví podle číselných ID uživatelů a
skupin.\\
\hspace*{1.5em}\texttt{-C /mnt/gentoo} - Extrahuje obsah do uvedeného adresáře.\\



\newpage
\subsection{\textsf{Kompilace}}
Součástí základního systému je Portage. Portage je nativní package manager pro Gentoo.
Obstarává kompilace, instalace, závislosti, synchronizace a~mnoho dalšího. V~této fázi instalace
je doporučené optimalizovat kompilace. Veškerá globální konfigurace je součástí \textit{/etc/portage/make.conf}.
Zároveň obsahuje přepínače pro překladače GCC, C++ a~Rust. Jejich správnému nastavení bude věnována následující podkapitola.

\subsubsection{\textsf{Flags}}
Pro nastavení GCC slouží proměnná \textit{CFLAGS} ve tvaru:\\ \hspace*{1.5em}\texttt{CFLAGS="flag1 flag2..."}\\Nastavení C++ je totožné, mění se pouze proměnná a~to na \textit{'CXXFLAGS'}.
Uvádíme-li, že nastavení jsou totožná, je to myšleno doslovně. Je doporučeno, aby přepínače u~překladačů byly při instalaci stejné. Níže je uvedena
vzorová konfigurace:\\
\hspace*{1.5em}\texttt{CFLAGS="-march=native -O2 -pipe"}\\
\hspace*{1.5em}\texttt{CXXFLAGS="-march=native -O2 -pipe"}

\hspace{-1.5em}\texttt{-march=native} - Použije nastavení a instrukční sady pro daný procesor.\\
\texttt{-O2} - Překladač se pokusí kompilovat rychleji na úkor paměti.\\
\texttt{-pipe} - Zrychluje kompilaci díky rourám, ty překladač vytvoří místo souborů.

\hspace{-1.5em}Proměnná pro Rust je \textit{RUSTFLAGS}:

\texttt{RUSTFLAGS="\$-C target-cpu=native opt-level=2"}

\hspace{-1.5em}\texttt{target-cpu=native} - Přizpůsobí kompilaci pro instalovaný procesor.\\
\texttt{opt-level=2} - Optimalizuje strojový kód, podobné -O2. %
\subsection{\textsf{MAKE Operace}}
Chceme-li, aby systém zůstal responzivní i~během kompilace, je potřeba omezit maxi-\\mální zatížení systému.
Přepínače \textit{make} omezující zátěž se zadávají do proměnné \textit{MAKEOPTS}:

\texttt{MAKEOPTS="-j5 -l5"}

\hspace{-1.5em}\texttt{-j} - Určuje maximální počet paralelních procesů.\\
\texttt{-l} - Nastavuje maximální vytížení procesoru.

\hspace{-1.5em}Důležitým faktorem pro volbu parametrů je, že každý proces zatěžuje mimo procesoru také paměť.

\newpage
\section{\textsf{Nový systém}}
Všechny následující kroky se již budou provádět v~novém systému.
Pro vstup do nového systému, jelikož se nejedná o~virtualizaci, je nutné zajistit přímý přístup k~fyzickým prostředkům.
Toho docílíme namontováním adresářů \textit{/sys}, \textit{/proc}, \textit{/dev} a~\textit{/run} na jejich \textit{/mnt/gentoo} protějšky. Chceme-li v~novém 
systému stejné DNS servery, zkopírujeme ještě jejich záznam: 

\texttt{cp /etc/resolv.conf /mnt/gentoo/etc} 

\hspace{-1.5em}Pokud se jedná o~symlink je potřeba přidat přepínač  \textit{--dereference}, \\
abychom zkopírovali soubor, ne odkaz na něj. 
\subsection{\textsf{Chroot}}
Do nového prostředí se dostaneme pomocí nástroje \textit{chroot} ve tvaru:

\texttt{chroot /mnt/gentoo chroot.sh}

\hspace{-1.5em}\texttt{/mnt/gentoo} - adresář s novým kořenem\\
\texttt{/bin/bash} - program/příkaz, který chceme v novém prostředí spustit

\hspace{-1.5em}Zároveň je nyní vhodné namontovat EFI oddíl na \textit{/efi} adresář.
\subsection{\textsf{Portage}}
Oficiální package manager pro Gentoo je Portage. Kromě instalací řeší závislosti, opravy, správu a~metadata balíčků.
Nejčastějším způsobem instalace je sestavení ze zdrojového kódu, avšak dnes již obsahuje i~mnoho binárních balíčků.
\subsubsection{\textsf{Základní příkazy}}
Software se instaluje pomocí příkazu:  

 \texttt{emerge kategorie/nazev} 
 
\hspace{-1.5em}Nevíme-li, některý z~výše uvedených parametrů, můžeme využít příkaz:

\texttt{emerge --search regulerni\_vyraz} 

\hspace{-1.5em}Příkaz pro jednoduchý update celého systému je:

\texttt{emerge -avuDN @world}

\hspace*{-1.5em}Uživatel by měl mít na paměti, že update může trvat až desítky hodin, viz přepínač \textit{-a}. %\textit{-v}

\subsubsection{\textsf{USE}}
Mezi hlavní výhody Portage patří USE Flags. Ty slouží ke konfiguraci sestavení a~instalace balíčků. Určují jaké funkce mají programy obsahovat a~podporovat.\\
Použijeme-li například vlajku \textit{X}, bude všechen software instalován s~podporou display serveru Xorg. Víme-li, že některé funkce nebudeme potřebovat, můžeme je odebrat pomocí znaménka mínus(pomlčky). Například \textit{-kde} odebere podporu grafických prostředí KDE.
Ne všechny balíčky přepínače USE podporují. Podporu přepínačů určují vývojáři.
Všechny použité přepínače se nachází ve složce \textit{/etc/portage/make.conf}. \\
Jejich seznam si můžeme vypsat pomocí příkazu:

\texttt{cat /etc/portage/make.conf | grep USE}

\hspace{-1.5em}Seznam všech přepínačů se nachází na webu:

\texttt{https://www.gentoo.org/support/use-flags/}\\

\hspace{-1.5em}Existují i~přepínače pro procesor a~grafický čip. Přepínače pro procesor se nastavují pomocí programu \textit{cpuid2cpuflags}:


\texttt{echo "*/* \$(cpuid2flags)" > /etc/portage/package.use/00cpu-flags}\\

\hspace{-1.5em}Přepínače pro grafický čip, v~případě potřeby, lze nastavit pomocí příkazu:

\texttt{echo "*/* VIDEO\_CARDS: *gpu*" > /etc/portage/package.use/00video-\\\hspace*{1.5em}cards}\\

\hspace{-1.5em}\textit{*gpu*} nahradíme podle následující tabulky:

\begin{table}[h]
	\centering
	\begin{tabular}{|c|c|c|}
		\hline
		Výrobce & Přepínač & poznámka \\
		\hline
		Nvidia & nvidia & -\\
		\hline
		Intel & intel & od 4. Generace \\
		\hline
		AMD & amdgpu radeonsi & od roku 2013/14 \\
		\hline
		Raspberry Pi & vc4 & - \\
		\hline
		QEMU&virgl&-\\
		\hline
		KVM&virgl&-\\
		\hline
		WSL&d3d12&- \\
		\hline
		Nvidia & nouveau & Open Source \tablefootnote{Všechny modely kromě architektur Maxwell, Pascal, Volta.}\\ 
		\hline
	\end{tabular}
\end{table}

\newpage
\subsubsection{\textsf{Licence}}
Od roku 2014 obsahují repozitáře Gentoo štítky s~licencemi balíčků. Pro lepší orientaci byla také zavedena proměnná \textit{ACCEPT\_LICENSE}. Licence byly rozděleny do skupin, které se deklarují v~\textit{/etc/portage/make.conf}:

\texttt{ACCEPT\_LICENSE="group"}

\hspace{-1.5em}Základní skupiny jsou:

\begin{table}[h]
	\centering
	\begin{tabular}{|c|c|c|}
		\hline
		Pro software & Popis & Svobodné \\
		\hline
		@GPL-COMPATIBLE & GPL kompatibilní & ANO \\
		\hline
		@FSF-APPROVED & Všechny licence schválené FSF & ANO \\
		\hline
		@MISC-FREE & Momentálně neschválené & ANO \\
		\hline		
		@EULA & Licence pro koncového uživatele & NE \\
		\hline
		@OSI-APPROVED & Schválené OSI & NE \\
		\hline
		@OSI-APPROVED-FREE & Schválené OSI & ANO \\
		\hline
		@OSI-APPROVED-NONFREE & Schválené OSI & NE \\
		\hline
		@BINARY-REDISTRUTABLE & Neschválené & - - - \\
		\hline

	\end{tabular}
\end{table}
\begin{table}[h]
	\centering
	\begin{tabular}{|c|c|}
		\hline
		Pro dokumenty & Popis \\
		\hline
		@FSF-APPROVED-OTHER & FSF schválené non-Software Licence \\
		\hline
		@MISC-FREE-DOCS & Neschválená \\
		\hline
	\end{tabular}
\end{table}
\begin{table}[h]
	\centering
	\begin{tabular}{|c|c|}
	\hline
	Nadskupiny & Popis \\
	\hline
	@FREE & @FREE-SOFTWARE $\cup$ @FREE-DOCUMENTS \\
	\hline
	@FREE-SOFTWARE & @FSF-APPROVED $\cup$ @OSI-APPROVED-FREE\\
	\hline
	@FREE-DOCUMENTS & @FSF-APPROVED-OTHER $\cup$ @MISC-FREE-DOCS\\
	\hline
	\end{tabular}
\end{table}
\hspace{-1.5em}Pokud uživatel chce instalovat software pod jakoukoliv licencí, může zvolit přepínač \textit{*}.
Jelikož se jedná o~limitující kategorii, skript nastaví hodnotu parametru na~\textit{*}. Důvodem je velké množství proprietárních ovladačů, nutných pro funkci fyzických prostředků.


\hspace*{-1.5em}V případě, že uživatel chce použít licenci pouze v~rámci balíčku, lze toho docílit pomocí záznamu v~\textit{/etc/portage/package.use/} ve tvaru:

\texttt{nazev/balicku licence}

\hspace{-1.5em}K propsání změn v~\textit{ACCEPT\_LICENSE} je nutné systém aktualizovat pomocí:

\texttt{emerge --ask --verbose --update --deep --newuse @world}


\newpage

\section{\textsf{Kernel}}\hypertarget{Kernel}{}
Jádro, kernel, je součástí většiny operačních systémů. Zajišťuje přístup k~fyzickým prostředkům počítače. Mezi hlavní funkce patří správa: procesorů, procesů, pamětí a~periferií. 

Program nikdy nekomunikuje přímo s~komponentou, místo toho komunikuje s~jádrem pomocí systémových volání, které pak předá instrukce komponentě. Výhodou je, že jádro může takto sbírat instrukce od více procesů a~následně je řadit podle plánovacího algoritmu, tzv. scheduler.

\subsection{\textsf{Scheduler}}
Scheduler pro linuxové jádro se nazývá {\bf C}ompletely {\bf F}air {\bf S}cheduler ({\bf CFS}). Účelem je poskytnout všem entitám stejný čas na procesoru. Entity zaštiťují jeden proces nebo skupinu procesů. Nejmenší možnou entitou je jeden jednovláknový proces, největší entita (skupina) může obsahovat všechny běžící procesy, taková entita je označena \textit{root\_task\_group}. 
\subsubsection{\textsf{Group scheduling}}
Skupiny mají přidělenou hodnotu \textit{cpu.shares}. Poměr hodnot \textit{cpu.shares} jednotlivých skupin určuje jejich čas na procesoru. Obsahuje-li skupina podskupiny, postup se opakuje. Dělení na skupiny není teoreticky nutné, avšak je žádoucí například u~systémů s~více uživateli.

\subsubsection{\textsf{Task scheduling}}
Po přidělení času (pod)skupinám je potřeba naplánovat jednotlivé procesy skupiny. Plánování se provádí pomocí hodnoty \textit{vruntime}. \textit{vruntime} je odvozen z~doby strávené na procesoru a~priority(NICE). Platí, že rostoucí priorita, klesající NICE, zpomaluje růst \textit{vruntime}. Procesy s~vyšší prioritou dostanou více prostoru. Procesy jsou podle \textit{vruntime} seřazeny do \textit{Red-black tree}. Jedná se o~upravený binární strom s~barevně označenými uzly. Strom vždy řadí menší hodnotu (v tomto případě \textit{vruntime}) vlevo. Barvy uzlů se průběžně mění, vždy však splňují čtyři podmínky:\\
1. Vkládaný uzel je červený.\\2. Každý uzel je černý nebo červený.\\3. Nikdy nesmí být dva červené uzly bezprostředně nad sebou.\\4. Cesty z~libovolných černých uzlů na konec obsahují stejný počet černých bodů. \\ 

\hspace*{-1.5em}Uvedené podmínky zaručují, že jakýkoliv takový strom s~$n$ uzly bude obsahovat \\maximálně $2\log(n+1)$ vrstev. Asymptotická složitost operací je $O(\log n)$.\\



\subsection{\textsf{Firmware}}
Firmware je kód běžící přímo na fyzické komponentě. Často je uložen v~paměti, která je součástí komponenty. V~případech, kdy tomu tak není, je potřeba jej nahrát z~kernelu operačního systému. Nejčastěji se jedná o~grafické karty, síťové a~zvukové karty. 

Firmware Linuxu uložen v~adresáři \textit{/usr/lib/firmware}, odkud je při zavádění načten do zařízení. Velkou část potřebného firmwaru obsahuje \textit{sys-kernel/linux-firmware}:

\texttt{emerge sys-kernel/linux-firmware}

\hspace{-1.5em}Pro procesory od výrobce Intel je potřeba stáhnout navíc mikrokód:

\texttt{emerge sys-firmware/intel-microcode}. 

\hspace{-1.5em}Mikrokód pro procesory AMD je součástí  \texttt{sys-kernel/linux-firmware}.
\subsection{\textsf{Initramfs}} \hypertarget{Initramfs}{}
{\bf Init}ial {\bf ram}-based {\bf f}ile {\bf s}ystem je malý filesystem. Cílem \textit{initramfs} je připravit minimální prostředí pro spuštění Init systému. Provádí načtení kernel modulů a~driverů pro zavedení důležitých filesystémů. Jakmile je vše potřebné připraveno, předá kontrolu Init systému.
Initramfs je plnohodnotné uživatelské prostředí, je tedy možné v~něm spouštět procesy. Standardním využitím této vlastnosti je záchranný systém nebo zadávání hesla pro rozšifrování disku. Méně standardním využitím je hraní hry Tetris. 

Generaci initramfs provádí \textit{Installkernel}, najde-li \textit{dracut} parametr v~souboru adresáře \textit{/etc/portage/}.

\hspace{-1.5em}Skript parametr vkládá do souboru \textit{installkernel}:

\texttt{echo "sys-kernel/installkernel dracut" | tee -a /\dots/installkernel}

\hspace{-1.5em}Úprava {Initramfs} %\hyperlink{Initramfs}
vyžaduje velké množství znalostí a~schopností, proto ji nebudeme více rozebírat. Podrobnější informace obsahuje Gentoo wikipedie. 


\subsection{\textsf{Linux Kernel}}
\subsubsection{\textsf{Architektura}}
Architektura linuxového jádra je monolitická, modulární. Je tedy možné dynamicky přidávat a~odebírat moduly. Linuxové jádro může tedy teoreticky obsahovat jen momentálně potřebné moduly. Jádro je sestaveno jako jeden blok, proto není potřeba mezi procesová komunikace jako u~mikrojádra. 
\subsubsection{\textsf{Kompilace}}
Skript nabízí dvě možnosti: lokální kompilaci a~předkompilovaný kernel. Kompilace jádra zabere mnoho času. Je-li čas na instalaci omezený, je doporučeno použít předkompilovaný kernel. Výhodou lokální kompilace je přizpůsobení podle přepínačů. 

\hspace{-1.5em}\texttt{select binsrc in "Kompilovat lokálně" "Použít předkompilovaný";do}}\\
\hspace*{1.5em}\texttt{break}\\
\texttt{done}\\
\texttt{if [ "\$binsrc" = "Kompilovat lokálně" ];then}\\
\texttt{\hspace*{1.5em}emerge --ask sys-kernel/gentoo-kernel}\\
\texttt{else}\\
\texttt{\hspace*{1.5em}emerge --ask sys-kernel/gentoo-kernel-bin}\\
\texttt{fi}\\
\newpage
\subsubsection{\textsf{fstab}}
V Linuxu se zařízení z~hlediska souborů dělí na bloková a~znaková. Bloková zařízení jsou především různá úložiště. Operace se provádějí v~blocích přes buffer. Znaková zařízení jsou zařízení s~krátkou odezvou, například: klávesnice, myš a~terminál. Operace se provádí ihned pomocí sekvence znaků(bytů).

K automatickému montování blokových zařízení slouží záznamy v~souboru \textit{/etc/fstab}. Struktura záznamu je:

\begin{table}[h]
	\begin{tabular}{|c|c|c|c|c|c|}
		\hline
		\texttt{zařízení} & \texttt{umístění} & \texttt{souborový systém} & \texttt{přepínač} & \texttt{Backup} & \texttt{fsck} \\
		\hline
		\texttt{/dev/sda3} & \texttt{/} & \texttt{XFS} & \texttt{defaults} & \texttt{0} & \texttt{2} \\
		\hline
		
	\end{tabular}
\end{table}
\hspace*{-1.5em}\texttt{zařízení} - device file z \textit{/dev adresáře}\\
\texttt{umístění} - lokace pro montování obsahu zařízení \\
\texttt{souborový systém} - souborový systém na zařízení\\
\texttt{přepínač} - určuje způsob montování\\
\texttt{backup} - zahrnutí zařízení do zálohy, boolean\\
\texttt{fsck} - {\bf f}ile{\bf s}ystem {\bf c}hec{\bf k} určuje v jakém pořadí jsou zařízení kontrolována\\

Z \hyperlink{Souborový systém}{kapitoly o~Souborovém systému} víme, že skript umožňuje výběr souborového systému pro \textit{root} oddíl, tento údaj je nutné předat mezi skripty \textit{instalator.sh} a~\textit{chroot.sh}. Skripty si proměnné předávají pomocí jednoduchých textových souborů.

\hspace*{-1.5em}Pro export proměnných:

\texttt{sudo echo \$filesystem | sudo tee -a /mnt/gentoo/filesystem} \\
\hspace*{1.5em}\texttt{sudo echo \$diskname1 | sudo tee -a /mnt/gentoo/diskname1} \\
\hspace*{1.5em}\texttt{sudo echo \$diskname2 | sudo tee -a /mnt/gentoo/diskname2} \\
\hspace*{1.5em}\texttt{sudo echo \$diskname3 | sudo tee -a /mnt/gentoo/diskname3} \\

\hspace*{-1.5em}Pro import proměnných:

\texttt{filesystem=\$(cat filesystem)}\\
\hspace*{1.5em}\texttt{diskname1=\$(cat diskname1)}\\
\hspace*{1.5em}\texttt{diskname2=\$(cat diskname2)}\\
\hspace*{1.5em}\texttt{diskname3=\$(cat diskname3)}\\

\hspace*{-1.5em}Tento postup není efektivní. Pro systémy s~jedním diskem se proměnné \textit{diskname} liší pouze v~jednom znaku. Pokud by však cílový uživatel chtěl přizpůsobit skript pro více disků je výše zmíněný přístup jednodušší na úpravu.
\newpage

\section{\textsf{Networking}}
Nyní je třeba zprovoznit síťová rozhraní. Potřebné balíčky se nainstalují pomocí příkazu:

\texttt{emerge --noreplace net-misc/netifrc}\\


\hspace{-1.5em}Zde skript umožňuje výběr ze dvou možností: DHCP a~Statická IP adresa. Obě možnosti jsou rozepsány v~následujících podkapitolách.

\hspace{-1.5em}Společnou částí je výběr síťového rozhraní. Vypsání všech dostupných síťových rozhraní provádí příkaz:

\texttt{ip link show | awk -F': ' '/\^~{[}0-9{]}+: / \{print \$2\}'}

\hspace{-1.5em}Vstup je poté uložen do proměnné \textit{iface}.	\\
Narozdíl od většiny ostatních proměnných zde není možnost výběru z~přednastavených hodnot. Důvodem je, že síťová rozhraní nemusí být připojená nebo mít k~dispozici drivery.
\subsection{\textsf{DHCP}}
DHCP provádí automatické nastavení IP adresy, výchozí brány a~DNS serveru pomocí služby DHCP.
Chceme-li, aby systém dostávál IP adresu z~DHCP serveru, je potřeba doinstalovat DHCP daemon:

\texttt{emerge net-misc/dhcpcd}

\hspace*{-1.5em}Konfigurace pro získání IP adresy se provádí v~souboru \textit{/etc/conf.d/net}. \\Struktura souboru vypadá následovně:

\texttt{config\_\$iface="dhcp"}

\subsection{\textsf{Statická IP adresa}}
Statická adresa se deklaruje v~souboru \textit{/etc/conf.d/net}. Je důležité mít na paměti, že DHCP nepřiděluje jen IP adresy, ale také nastavuje adresy výchozí brány a~DNS serveru. Syntaxe pro zápis statické adresy je:

\texttt{config\_\$iface="\$ip netmask \$mask brd \$brd"}\\
\hspace*{1.5em}\texttt{routes\_\$iface="default via \$routerIP"}
\subsubsection{\textsf{Broadcast Calculator}}
Pro zjednodušení konfigurace statické IP adresy je součástí repozitáře i~jednoduchý program na výpočet adresy broadcastu, Broadcast Calculator. Použití programu je:

\texttt{brdcal IP\_ADRESA MASKA\_PODSITE}


\hspace*{-1.5em}Výstupem pak je IP adresa broadcastu pro danou podsíť. \\


Aby byla zajištěna spolehlivost a~následná kompatibilita se skriptem, není možné použít vyšší programovací jazyky (Python, C\#, Java...), systém je nemusí podporovat. Z~toho důvodu je celý program napsaný v~jazyce C a~za použití pouze tří knihoven: \textit{stdio.h}, \textit{stdlib.h} a~\textit{string.h}. Asymptotická složitost algoritmu je $O(1)$.

\newpage
\section{\textsf{Bootloader}}

Hlavním úkolem bootloaderu, zavaděče, je nahrát \hyperlink{Kernel}{jádro operačního systému} a~\hyperlink{Initramfs}{Initramfs} do paměti RAM. Dělí se na bootloadery první fáze a~druhé fáze.
\subsection{\textsf{První fáze}}
Mluvíme-li o~bootloaderu první fáze, máme téměř vždy na mysli firmware základní desky. Nejčastěji UEFI nebo BIOS. Jelikož většina počítačových systémů používá UEFI, nebudeme zde více rozebírat BIOS ani MBR.

\hspace{-1.5em}Bootloader ihned po zapnutí provede {\bf P}ower {\bf O}n {\bf S}elf {\bf T}est. Tím se zajistí, že všechny komponenty pracují správně. V~opačném případě se bootování zruší a~UEFI, často pomocí reproduktoru, vrátí chybu. Je dobré zmínit, že úspěšný POST neznamená plně funkční komponenty. Následně začne hledat EFI oddíl, ve skriptu proměnná \textit{diskname1}. Z~oddílu se pokusí načíst bootloader druhé fáze, kterému poté předá řízení.

\subsection{\textsf{Druhá fáze}}
Úkolem bootloaderu druhé fáze je načíst operační systém do paměti RAM. Nejpoužívanějším bootloaderem druhé fáze pro Linux je GRUB. Většina bootloaderů, včetně GRUBu, jsou technicky zároveň i~bootmanagery, jelikož nabízí možnost načtení různých programů. Kromě načítání operačních systémů dokáží moderní bootloadery načíst testery, hry, jiné bootloadery a~téměř libovolný jiný program. Skript provádí instalaci zavaděče GRUB podle GPT schématu:\\

\texttt{echo 'GRUB\_PLATFORMS="efi-64"' | tee -a /etc/portage/make.conf}\\
\hspace*{1.5em}\texttt{emerge --ask sys-boot/grub}\\
\hspace*{1.5em}\texttt{grub-install --efi-directory=/efi}\\
\hspace*{1.5em}\texttt{grub-mkconfig -o /boot/grub/grub.cfg}\\

\newpage
\section{\textsf{Služby}}
Součástí skriptu je možnost instalace a~zprovoznění užitečných služeb. Služby pevně spjaté se systémem, tj. spravující systém, jsou nainstalovány a~zprovozněny již v~rámci \textit{chroot.sh} skriptu. Ostatní služby mají skripty uložené v~adresáři \textit{Services} a~je tudíž potřeba je spustit manuálně. Seznam služeb a~jejich využití je blíže popsán v~následujících podkapitolách. 

\subsection{\textsf{SecureShell}}
\textit{SecureShell}, častěji označován zkratkou \textit{SSH}, je program umožňující vzdálenou správu počítače pomocí textového rozhraní. \textit{SecureShell} je součástí základní sady nástrojů Gentoo, proto není třeba jej instalovat zvlášť. Skript tedy službu přidá do seznamu pro spuštění při startu: 

\texttt{rc-update add sshd default}
\subsection{\textsf{Sysklogd}} 
\textit{Sysklogd} je daemon pro logovací službu \textit{Syslog}. Provedena je pouze instalace a~přidání do seznamu Init systému(OpenRC). Služba pracuje podle výchozí konfigurace. V~případě nutnosti konfigurace je třeba upravit soubor \textit{/etc/syslog.conf}.

\subsection{\textsf{Cronie}}

\textit{Cronie} je moderní plánovač úloh postavený na službě \textit{Cron}. \textit{Cronie} umožňuje plánování na základě intervalu nebo data. Veškeré úlohy se plánují ve souborech \textit{crontab}. \textit{Crontab} soubor pro systémové úlohy je umístěn v~\textit{/etc/crontab}. Syntaxe pro \textit{crontab}:\\

\texttt{*$_{1}$ *$_{2}$ *$_{3}$ *$_{4}$ *$_{5}$ /umisteni\_skriptu}

\begin{table}[h]
	\centering
	\begin{tabular}{|c|c|}
		\hline 
		Lokace&Popis \\
		\hline
		*$_{1}$&  Minuta : 0-59 \\
		\hline
		*$_{2}$&  Hodina : 0-23 \\
		\hline
		*$_{3}$&  Den v měsíci : 1-31 \\
		\hline
		*$_{4}$&  Měsíc v roce : 1-12 \\
		\hline
		*$_{5}$&  Den v týdnu : 0-6 \\
		\hline
		*/cislo&  Označuje periodu dle umístění. \\
		\hline
		*&  Asterisk označuje jakoukoliv hodnotu. \\
		\hline
	\end{tabular}

\end{table}
\subsection{\textsf{WPA Supplicant}}
Připojení k~WIFI síti, narozdíl od Ethernetu, vyžaduje dodatečnou konfiguraci. Častým nástrojem pro připojení k~WIFI sítím je \textit{wpa\_supplicant}. Po kompilaci je potřeba vytvořit konfigurační soubor pro síťové rozhraní. Konfigurační soubory jsou umístěny v~adresáři \textit{/etc/wpa\_supplicant/}.\\ Skript vytváří konfigurační soubor podle názvu rozhraní:\\

\texttt{echo """ctrl\_interface=/run/wpa\_supplicant
	update\_config=1 """ > \hspace*{1.5em}/etc/wpa\_supplicant/wpa\_supplicant-\$\{wlancard\}.conf}\\

\hspace{-1.5em}Pro připojení~k samotné síti je potřeba název (SSID)~a heslo (PSK). Záznam se generuje pomocí \textit{wpa\_passphrase} přesměrovaným do konfiguračního souboru: \\

\texttt{wpa\_passphrase \$\{ssid\} \$\{psk\} >> \\\hspace*{1.5em}/etc/wpa\_supplicant/wpa\_supplicant-\$\{wlancard\}.conf}\\

\hspace{-1.5em}První připojení provede skript automaticky, nicméně je dobré zmínit strukturu příkazu pro opětovné připojení:\\

\texttt{wpa\_supplicant -i nazev\_rozhrani -c /umisteni/konf\_souboru}

\hspace*{-1.5em}Pro automatické připojení při zapnutí terminálu lze příkaz vložit do souboru '\textit{.bashrc}'.
\subsection{\textsf{MariaDB}}
\textit{MariaDB} patří k~nejpoužívanějším databázovým serverům. Hlavní výhodou oproti konkurenci je licence \textit{GNU GPLv2} a~plná kompatibilita s~\textit{MySQL}. Skript provede kompilaci včetně základní konfigurace.\\

\texttt{read -p "heslo pro MariaDB: " \$password}\\
\hspace*{1.5em}\texttt{emerge dev-db/mariadb}\\
\hspace*{1.5em}\texttt{mysql\_secure\_installation <<-YO}\\
\hspace*{3em}\texttt{\$password}\\
\hspace*{3em}\texttt{\dots}\\
\hspace*{1.5em}\texttt{YO}

\newpage
\section{\textsf{Závěr}}
Cílem maturitní práce bylo usnadnit instalaci a konfiguraci operačního systému Gentoo. Teoretická část zároveň slouží jako dokumentace kódu a příručka pro nové uživatele. 

Za osobní přínosy maturitní práce považuji zejména zdokonalení v programovacích jazycích Bash a C. V průběhu tvorby práce jsem využil verzovací software Git a web Github. Teoretická část maturitní práce je psána v LaTeXu a sestavena pomocí nástroje TeXstudio. Velkou pomocí byla Gentoo wiki, která je spravována dobrovolníky z~Gentoo komunity.

Cíl práce je zaměřen do budoucna, tudíž není možné dopředu předpovědět splnění vytyčených cílů. Nicméně zpřístupnění zdrojového kódu pod BSD licencí usnadní adopci a~spolupráci na jeho vývoji a~úpravě v budoucnosti. 
\newpage
\section{\textsf{Zdroje}}
\begin{enumerate}
	\item KERRISK, Michael. Linux manual page. Online. Man7. 2025. Dostupné z: https://man7.org/linux/man-pages/man8/fdisk.8.html. [cit. 2025-03-15].
	\item /etc/portage/make.conf. Online. Gentoo Wiki. C2001–2025. Dostupné z:\\ https://wiki.gentoo.org/wiki//etc/portage/make.conf. [cit. 2025-01-19].
	\item GLEP 23. Online. Gentoo. C2001–2025. Dostupné z: \\https://www.gentoo.org/glep/glep-0023.html. [cit. 2025-02-25].
	\item License Groups. Online. Gentoo Wiki. C2001–2025. Dostupné z: \\https://wiki.gentoo.org/wiki/License\_groups. [cit. 2025-02-25].
	\item Jádro operačního systému. Online. In: Wikipedia: the free encyclopedia. San Francisco (CA): Wikimedia Foundation, 2001-. Dostupné z:\\https://cs.wikipedia.org/wiki/J\%C3\%A1dro\_opera\%C4\%8Dn\%C3\%ADho\_\\syst\%C3\%A9mu. [cit. 2024-09-11].
	\item EEVDF. Online. In: Wikipedia: the free encyclopedia. San Francisco (CA): Wikimedia Foundation, 2001-, 2024-6-21. Dostupné z:\\ https://en.wikipedia.org/wiki/Earliest\_eligible\_virtual\_deadline\_first\_scheduling. [cit. 2024-09-11].
	\item EEVDF Notes [@Kuanch]. Online. 2024, 8 August 2024. Dostupné z: HackMD, https://hackmd.io/@Kuanch/eevdf. [cit. 2025-03-12].
	\item EEVDF Scheduler Merged. Online. Phoronix. Dostupné z:\\ https://www.phoronix.com/news/Linux-6.6-EEVDF-Merged. [cit. 2025-01-05].
	\item Completely Fair Scheduler. Online. In: Wikipedia: the free encyclopedia. San Francisco (CA): Wikimedia Foundation, 2001-. Dostupné z:\\ https://en.wikipedia.org/wiki/Completely\_Fair\_Scheduler. [cit. 2025-01-05].
	\item KHAN, Imran. CFS Group Scheduling. Online. Oracle Linux Blog. 2024. Dostupné z: https://blogs.oracle.com/linux/post/cfs-group-scheduling. \\{[cit. 2025-01-05]}.
	\item Introduction To Algorithms. Online. MIT OpenCourseWare. Dostupné z:\\ https://ocw.mit.edu/courses/6-046j-introduction-to-algorithms-sma-5503-fall\\-2005/resources/lecture-10-red-black-trees-rotations-insertions-deletions.\\ {[cit. 2025-01-05]}.
	%https://www.cs.usfca.edu/\~galles/visualization/RedBlack.html
	\item MORRIS, John. Data Structures and Algorithms. Online. EECS University of Michigan. Dostupné z: https://www.eecs.umich.edu/courses/eecs380/ALG/\\red\_black.html. [cit. 2025-01-03].
	\item Red-black tree. Online. In: Wikipedia: the free encyclopedia. San Francisco (CA): Wikimedia Foundation, 2001-. Dostupné z: https://en.wikipedia.org/wiki/\\Red\%E2\%80\%93black\_tree. [cit. 2025-03-16].
	\item About Firmware. Online. Linux From Scratch. Dostupné z: \\https://www.linuxfromscratch.org/blfs/view/svn/postlfs/firmware.html.\\ {[cit. 2025-02-09]}.
	\item Initramfs. Online. Gentoo Wiki. C2001–2025. Dostupné z:\\ https://wiki.gentoo.org/wiki/Initramfs. [cit. 2025-02-09].
	\item HAND, David [@ptolemarch]. Linux initramfs for fun, and, uh... Online. 2018. Dostupné z: Youtube, https://www.youtube.com/watch?v=KQjRnuwb7is.\\ {[cit. 2025-02-09]}.
	\item Custom Initramfs. Online. Gentoo Wiki. C2001–2025. Dostupné z: \\https://wiki.gentoo.org/wiki/Custom\_Initramfs. [cit. 2025-02-09].
	\item Mikrojádro. Online. In: Wikipedia: the free encyclopedia. San Francisco (CA): Wikimedia Foundation, 2001-. Dostupné z: https://cs.wikipedia.org/wiki/\\Mikroj\%C3\%A1dro. [cit. 2025-01-11].
	\item Hybridní jádro. Online. In: Wikipedia: the free encyclopedia. San Francisco (CA): Wikimedia Foundation, 2001-. Dostupné z:\\ https://cs.wikipedia.org/wiki/Hybridn\%C3\%AD\_j\%C3\%A1dro.\\ {[cit. 2025-01-11]}.
	\item Monolitické jádro. Online. In: Wikipedia: the free encyclopedia. San Francisco (CA): Wikimedia Foundation, 2001-. Dostupné z: https://cs.wikipedia.org/wiki/\\Monolitick\%C3\%A9\_j\%C3\%A1dro. [cit. 2025-01-11].
	\item Zařízení (soubor). Online. In: Wikipedia: the free encyclopedia. San Francisco (CA): Wikimedia Foundation, 2001-. Dostupné z: https://cs.wikipedia.org/wiki/\\Za\%C5\%99\%C3\%ADzen\%C3\%AD\_(soubor). [cit. 2025-01-12].
	\item /etc/fstab. Online. Gentoo Wiki. C2001–2025. Dostupné z:\\ https://wiki.gentoo.org/wiki//etc/fstab. [cit. 2025-02-23].\newpage
	\item High-level programming language. Online. In: Wikipedia: the free encyclopedia. San Francisco (CA): Wikimedia Foundation, 2001-. Dostupné z:\\ https://en.wikipedia.org/wiki/High-level\_programming\_language.\\ {[cit. 2025-01-13]}.
	\item Bootloader. Online. In: Wikipedia: the free encyclopedia. San Francisco (CA): Wikimedia Foundation, 2001-. Dostupné z:\\ https://en.wikipedia.org/wiki/Bootloader. [cit. 2025-02-24].
	\item Sysklogd. Online. Gentoo Wiki. C2001–2025. Dostupné z:\\ https://wiki.gentoo.org/wiki/Sysklogd. [cit. 2025-03-11].
	\item Cron. Online. Gentoo Wiki. C2001–2025. Dostupné z:\\ https://wiki.gentoo.org/wiki/Cron. [cit. 2025-03-14].
	\item MariaDB. Online. In: Wikipedia: the free encyclopedia. San Francisco (CA): Wikimedia Foundation, 2001-. Dostupné z: \\https://en.wikipedia.org/wiki/MariaDB. [cit. 2025-03-15].
\end{enumerate}
\end{document}
